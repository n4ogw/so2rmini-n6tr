%Copyright Larry Tyree, N6TR, 2011,2012,2013,2014,2015.

%This file is part of TR log for linux.

%TR log for linux is free software: you can redistribute it and/or
%modify it under the terms of the GNU General Public License as
%published by the Free Software Foundation, either version 2 of the
%License, or (at your option) any later version.

%TR log for linux is distributed in the hope that it will be useful,
%but WITHOUT ANY WARRANTY; without even the implied warranty of
%MERCHANTABILITY or FITNESS FOR A PARTICULAR PURPOSE.  See the
%GNU General Public License for more details.

%You should have received a copy of the GNU General
%    Public License along with TR log for linux.  If not, see
%<http://www.gnu.org/licenses/>.

\documentclass[12pt]{article}
\usepackage{graphicx}
\usepackage{amsmath}
\usepackage{amssymb}
\usepackage{url}
\usepackage[margin=.9in]{geometry}
\usepackage{makeidx}
\makeindex
\begin{document}
\title{A TR log linux port}
\author{
Kevin E. Schmidt, W9CF\\
6510 S. Roosevelt St.\\
Tempe, AZ 85283 USA\\
}
\date{}
\maketitle
\vskip 2cm
\tableofcontents
\vskip 2cm
\section{Introduction}
Read the standard legal disclaimer in section \ref{disclaim}.
\vskip 0.5cm

TR log or TR was written by Tree, Larry Tyree, N6TR. It was written for PCs
running DOS. It continued to work fine running versions of Windows based
on DOS, i.e. Windows 95, 98, ME etc., but does not run without
modifications under modern
multitasking multiuser versions of Windows that are based on Windows NT,
and obviously will not run natively on unix based operating systems like
linux or Mac OS X.

For many years I have run TRlog under linux using dosemu, a program that
runs a version of DOS in linux. While this works for the most part,
the timing for sending CW requires taking care that dosemu gets high
priority, and even so occasionally poor cw is sent. Getting a sound card
working is difficult, and the whole thing is a ugly.

In November 2011, I asked Tree if he would be willing to let me try to
port TR log to linux. He very generously sent me the complete source
code to TR log version 6.76, along with the comment ``I suppose I can
zip up the source code -- but I think you are nuts!''

TR log is written in Pascal. In particular, I learned that it
is written in Borland Pascal 7.0. Happily, Free Pascal,\index{Free Pascal
Compiler}
available at
\url{<http://freepascal.org>},
is an excellent multiplatform pascal compiler that supports
nearly all the Borland Pascal 7.0 features. This made porting
the code much less painful.

\section{Goal}

Please realize, that my goal was to port the code to run on modern
hardware with a multitasking multiuser operating system while doing
the least
amount of work. Therefore I decided to run TR log in a terminal window
that would look and work like the DOS version of TR.

I have
added few new
features. I have tried to remove as few features of the DOS based
TR log as possible. 

\section{My tests}
I have tested this code only on my linux systems which all run a recent
version of Slackware, mostly Slackware 13.37.0, and all run the fairly
lightweight xfce desktop environment. There are many features that I 
have not tested. Most of these should work since I have not changed Tree's
code intentionally.

\section{Quick start}
\begin{enumerate}
\item
Download the latest trlinux-rN.NN.txz, where the N.NN is the version number.
The TR linux web site is \url{<http://www.kkn.net/trlinux/>}\index{Web Site},
\item
Untar this in a convenient location at the command line
\begin{verbatim}
tar -xJf trlinux-rN.NN.txz
\end{verbatim}
\item
In the trlinux-rN.NN/files subdirectory find the Xresources file\index{
.Xresources file}. Include
these lines in the .Xresources file in your \$HOME directory, (or
create this file if it doesn't exist), for example type:
\begin{verbatim}
cd trlinux_rN.NN/files
cat Xresources >> $(HOME)/.Xresources
\end{verbatim}
To make these changes take effect during this X-windows session
you can run at the command line
\begin{verbatim}
cd
xrdb -merge .Xresources
\end{verbatim}
It will be automatically reloaded when you log in.
\item
Test that this works with the commands
\begin{verbatim}
xterm -class DosTerm
\end{verbatim}
and
\begin{verbatim}
xterm -class BigDosTerm
\end{verbatim}
Feel free to change the font\index{Changing the font}
specified in the .Xresources file for these
to get a convenient size. A list of font names you can try is given by typing
\begin{verbatim}
fc-list :scalable=true:spacing=mono: family
\end{verbatim}

{\em It is absolutely necessary that
the DosTerm be 80 columns and 25 rows,
and the BigDosTerm 80 columns and 50 rows. }
You can check this with the commands
\begin{verbatim}
set |grep COLUMNS
set |grep LINES
\end{verbatim}
the first should print COLUMNS=80 and the second LINES=25 or LINES=50 depending
on whether you are in a DosTerm or a BigDosTerm. If these do not
match, you likely have a font that is too big to allow the xterm its
full size. In that case change the faceSize\index{Changing the font size}
value in the .Xresources file.

\item
All of the files that you need to run TRlog are in the subdirectory
trlinux-rn.nn/log.
Place these in any convenient directory (or leave them where they are).
I will call this directory path-to-trlog.
The executable name is then
\begin{verbatim}
path-to-trlog/trlog
\end{verbatim}
since {\bf tr} is already a system utility in
linux/unix. Feel free to put the directory path-to-trlog in your PATH
environment variable to reduce typing.
\item
Start a DosTerm  with
\begin{verbatim}
xterm -class DosTerm
\end{verbatim}
Change to a convenient empty directory, and start Trlog, for example
\begin{verbatim}
cd
mkdir trtest1
cd trtest1
path-to-trlog/trlog
\end{verbatim}
You should get the prompt
\begin{verbatim}
Do you want to step through setting up a new contest? (Y/N) :
\end{verbatim}
answer Y, and choose a contest\index{New contest prompt}
-- try CQWW to test things out. 
Give your call when prompted, and follow the directions more or less
as in DOS TR log to do a basic set up.
\item
You should have a trlog screen in the xterm.
Try logging a few contacts. 
\item
If this works you're essentially ready to go. See the later
sections for more details, and how to set up a multi network if you
want to connect more than one instance of trlog together.
\end{enumerate}

\section{Setting up TR log on linux}
TR under linux should be run as a normal user. I run it under my usual
login account. That user/account should have permission to read from and
write to any devices you use to interface to your rig.

Since linux is case sensitive, Tree wrote the file names in the
source in all capitals, and all the commands in the LOGCFG.DAT file
are converted to uppercase before interpreting them, all file names used
by TR should be in capital letters. So,
for example, the default file names will be
LOGCFG.DAT, LOG.DAT, LOG.TMP, RESTART.BIN, BANDMAP.BIN etc.


\section{Window manager hot keys}
Most window managers\index{Window Manager hot keys}
grab various function key presses before they
are sent to applications. Since TRlog uses all the combinations of
function, ctrl-function, alt-function, shift-function, you should turn
off the window manager hot keys (or at least the ones you want to use
in TRlog). How to do this varies according to the window manager you run.
For example for xfce go to Settings-$>$window manager-$>$Keyboard, and clear
the shortcuts that you want to use under TRlog. In most window managers,
you can set up different profiles so that you can have one profile for
normal computing when you want the hot keys to work, and another when you
want to run TRlog.

\section{PC speaker}
Under linux, the PC speaker\index{PC Speaker}
is accessed through /dev/console which, for
security reasons, can only be read or written by root. If you
want to use the PC speaker for beeps like in DOS, you must run the linux trlog 
executable as the root user.
As an alternative, you can use the beep soundcard enable command described
later. Nothing else should require root permissions!

\section{New configuration commands}
The main change to the LOGCFG.DAT is that since the program now goes through
the standard drivers, the serial\index{Serial ports}
and parallel port\index{Parallel ports} devices must use linux
device names. Whenever a port number was given in the DOS TR configuration
command, a full device path needs to be given under linux. {\em These
are case sensitive and must match the linux device names.}
Here is an example of the lines that I use for radio control and keying
of my Elecraft K2 using a usb serial port adapter. The adapter comes
up under linux as the device /dev/ttyUSB0.
\begin{verbatim}
keyer output port = serial /dev/ttyUSB0
radio one type = k2
radio one control port = serial /dev/ttyUSB0
radio one baud rate = 4800
\end{verbatim}

A similar set up on my shack computer which has hardware
serial and parallel ports uses:
\begin{verbatim}
keyer output port = parallel /dev/parport0
radio one type = k2
radio one control port = serial /dev/ttyS0
radio one baud rate = 4800
paddle input = parallel /dev/parport0
\end{verbatim}

The serial port device names are those set up by the kernel.
The parallel port names are the device names reported by
libieee1284. Usually these will be /dev/parport0, /dev/parport1, etc.

\section{Port permissions}
You will need appropriate
permission\index{Serial port permissions}\index{Parallel port permissions}
to read and write to the ports.
The simplest way is to add the username of the account to the groups
that own the ports. For example, when plugging in a USB to serial
port adapter on my Slackware 13.37.0 system, the device is /dev/ttyUSB0
which has permissions from ls -l /dev/ttyUSB0
\begin{verbatim}
crw-rw---- 1 root dialout 188, 0 Feb 22 22:12 /dev/ttyUSB0
\end{verbatim}
which is owned by the dialout group\index{dialout group}.
I added my user name (w9cf) to
that entry in /etc/group using the command\index{Adding user to group}
\begin{verbatim}
sudo gpasswd -a w9cf dialout
\end{verbatim}
You must log out and log back in for this to take effect.

\section{K1EL Winkeyer support}
Initial limited support of the K1EL Winkeyer\index{Winkeyer} has been added.
I have done limited testing with
with my Winkeyer2 v2.3 chip. It should support two radios, both with
ptt, and a single sidetone if desired.

Older version 1 chips are supported, but since they have only two output
pins, I have chosen to use these for keying the two radios. No PTT or
sidetone is available with these chips. Since you can order a WinKeyer2
chip from K1EL for 11 US dollars, I decided it was not worth more than
a few minutes of my time to support the older chips.

There is a new configuration
command. Add a line like:
\begin{verbatim}
winkeyer port = serial /dev/ttyUSB0
\end{verbatim}
to your LOGCFG.DAT file\index{WINKEYER PORT},
substituting your serial port device that
is connected to the Winkeyer for /dev/ttyUSB0 above.

Most of the standard TR commands should still work. Some things that
are currently different or should be mentioned (there are probably others)
\begin{itemize}
\item
The Winkeyer paddle speed is locked to the trlog speed.
\item
The speed pot on the Winkeyer is ignored. (This is untested since I
do not have a speed pot hooked to my Winkeyer2 v2.3 chip.)
\item
You cannot have a separate paddle sidetone frequency. (Actually, none
of the Winkeyer sidetone stuff is tested since I do not have a speaker
connected to my Winkeyer2 v2.3 chip.)
\item
You cannot change the weight in a message using ctrl-x, ctrl-y.
\item
Speed changes with ctrl-f and ctrl-s should work fine.
\item
None of the special length dots and dashes work. That is do not
use ctrl-p, ctrl-q, ctrl-\textbackslash, ctrl-v, ctrl-l, ctrl-e, ctrl-d, ctrl-k
ctrl-n, ctrl-o, or - in a message.
\item
During CW, the Winkeyer handles PTT. It does not handle push to talk
in other modes. This can be added if there is sufficient interest.
The trlog PTT setting are approximately mapped to Winkeyer settings.
So Paddle PTT Hold Count is converted to the nearest Winkeyer hang time.
PTT Turn On Delay is converted into the nearest Winkeyer leadin.
\item
Footswitch CW grant does not work with the Winkeyer. This could be
added if there is sufficent interest.
\item
Winkeyer PTT tail time has been fixed at 20 milliseconds.
\item
Choosing the simulator will disable the Winkeyer and use Tree's original
CPU keyer to generate the simulated CW.
\end{itemize}

To add Winkeyer support, I encapsulated Tree's serial and parallel
port keyer object as a class.  This made adding Winkeyer support
straightforward, however, be aware that this could easily have added
bugs to the original serial and parallel port keyers.

\section{Bug fixes and upgrades}
As mentioned in the introduction, the code base is TRlog version 6.76.
I have fixed one bug and added support for newer CTY.DAT with specific
call signs.
\begin{itemize}
\item
I have fixed the 160m bandmap bug -- see
\url{http://lists.contesting.com/_trlog/2003-10/msg00092.html}.
\item
You should use the CTY.DAT file for TRlog version 6.90 and later which
flags individual calls with an equal sign. For example the entry =W9CF(3)[6]
shows that I am in CQ zone 3 and ITU zone 6 rather than the default
W9 zones 4 and 8. This same entry in the
CTY.DAT file for earlier versions is W9CF(3)[6] without the equal sign. The
problem with this older version is that W9CF is treated as a prefix
and W9CFA through W9CFZ would be
flagged as being in zones 3 and 6 too. With the equal sign in front
of my call there has to be an exact match.
\item
The dvp commands can be used to play files\include{DVP commands}. Currently
these must be at 44100 Hz stereo, i.e. cd quality
audio\include{DVP file format}. They can
be wave files, or, if your distribution has compiled ogg support
into libsndfile, they can be ogg files. (You can take your
wave file temp.wav and encode it with oggenc temp.wav then rename
it to CQF1.DVP etc., or just rename temp.wav to CQF1.DVP.) The sound
uses the ALSA (advanced linux sound architecture)
libraries\index{ALSA libraries} so
the kernel must have ALSA support. By default it opens the hw:0,0 device
so the first soundcard will be used. It must support 44100 Hz stereo.
You can change the device with the configuration file command like
\begin{verbatim}
SOUNDCARD DEVICE = plughw:0,0
\end{verbatim}
which would open the alsa software plugin for the same device.
Push to talk should now work with the dvp commands. It is implemented
in all current push to talk methods YCCC SO2R box, Winkeyer,
and serial/parallel port push to talk.
\item
If you are not using the dvp files, you can give the command
\begin{verbatim}
beep soundcard enable = true
\end{verbatim}
which will play the pc speaker beeps\index{BEEP SOUNDCARD ENABLE}
through the sound card.
Again this requires ALSA and 44100 Hz stereo support.
Note that, just like the original pc speaker,
no shaping is done, so the clicks when using this as a sidetone or 
with the simulator get annoying quickly.

\item
Since parallel ports are getting rare, and USB to parallel adaptors
don't work very well for non printer applications, I added paddle
input on serial ports\index{Paddle input on serial ports}.
This violates the RS-232 standard which defines
the two logic levels as positive and negative voltages, while the
paddle will only key a positive voltage to ground. But, according
to the ARRL
handbook, none of the serial port chips used actually require
a negative voltage on their inputs to get the correct logic level.
Therefore, I have set up the clear to send (CTS) input as the left
paddle (normally dot) input, and data set ready (DSR) input as the
right paddle (normally dash) input. I pull these inputs to +12 volts through
a 10K ohm resistor (the RS-232 standard requires between 3 and 25 volts),
and the paddle keys those inputs to ground. On a DB9 connector they
are pins 6 (DSR) and 8 (CTS). Ground is pin 5. It is then possible
to have radio control, radio keying, push to talk, and paddle input
on a single serial port. I use a USB serial port with the line
\begin{verbatim}
paddle port = serial /dev/ttyUSB0
\end{verbatim}
in LOGCFG.DAT. Probably the two easiest places to get the pull-up voltage
are from the rig power supply or use the +5 volt output of a USB port.

I have recently noticed that using the serial paddle input with
USB to serial devices can affect the CW quality. On my brief experiments,
a USB to serial converter using an FTDI chip slowed down the CW substantially
above about 25 wpm. A different converter using a Prolific chip worked
fine for CW up to about 35 wpm.

\item
Since the serial ports are using the linux drivers, you should
be able to use
any baud rate\index{Baud rates}
supported by your hardware (up to at least 115200 baud
for modern serial ports). Since the linux kernel can buffer up
to 4096 bytes, the only caveat is that trlog will have a maximum throughput
on the serial ports set by the emulated DOS interrupt which
occurs about every 1.7 milliseconds. I believe this should be adequate for
any rig.

\item
Winkeyer support has been added.

\item
Ultimatic mode has been added\index{Ultimatic mode}.
This is free with Winkeyer, it now also
works with the original CPU keyer.
Add
\begin{verbatim}
curtis keyer mode = u
\end{verbatim}
to your LOGCFG.DAT or use the control-j menu.

\item
The shift key enable command can be true and false as in DOS TR. I have
added\index{SHIFT KEY ENABLE additions}
\begin{verbatim}
SHIFT KEY ENABLE = ALTSHIFT
\end{verbatim}
In this mode, you need to first hold either alt key before pushing
left or right shift to tune the radio or RIT. Shifted characters like
the question mark then do not have to be remapped.

\end{itemize}

\section{Things that no longer work}
\begin{itemize}

\item
DVP record/backtrace etc. These can be added, but they are
low priority for me.
\item
DVP PATH has not been implemented. Place the audio files in the
current directory with LOGCFG.DAT.

\end{itemize}


\section{Packet using the shell device}
The preferred way to use packet\index{Packet shell device}\index{Telnet packet}
is now with the shell device.
Use the lines
\begin{verbatim}
PACKET PORT = SERIAL shell
EIGHT BIT PACKET PORT = TRUE
\end{verbatim}
If you now hit control-b, you will get the usual packet window, but
it will now show a shell prompt. You can simply type a command like
\begin{verbatim}
telnet n7az.net
\end{verbatim}
login with your call sign, and type sh/dx or just wait for spots.
If you have other software that provides packet filtering, etc., you
can use that.

\section{Networking with the ncat device}
My Slackware distribution comes with ncat installed as a part of the
nmap package, \url{<http://nmap.org/ncat/>}.  I assume, therefore,
that most other distributions will have ncat
available\index{ncat for networking}. The network
is set up so each trlog instance talks to the next, forming
a loop just like the original serial port network. That way most of
Tree's code can be used without change.

{\em Security disclaimer:} I implemented this without any encryption.
You should therefore trust the other users on your local network,
and the port(s) you choose for the communication should be behind
a firewall that does not expose them to the rest of the internet.
If you have a real need to use secure sockets, because of networking
between distant stations, let me know and I can try to get that
set up. 

To use ncat networking, the ncat command should be in your PATH.
Add a line like:
\begin{verbatim}
MULTI PORT = SERIAL ncat;12002;shack.myhouse.org:12001
\end{verbatim}
or
\begin{verbatim}
MULTI PORT = SERIAL ncat;12001;192.168.14.27:12001
\end{verbatim}

The form is
\begin{verbatim}
MULTI PORT = SERIAL ncat;localport;nextmachine_ip_address:nextmachine_port
\end{verbatim}

If all the trlog instances are running on different computers, then
all the ports can be the same number. This should be an unused port
number greater than 1024 (since ports below 1024 are privileged).
You can check the file /etc/services for the well known port numbers
and make sure you are not using any of those.

Trlog will listen on the localport and will send to nextmachine\_port
on the machine with ip address nextmachine\_ip\_address. Be sure
to notice the use of both semicolons (to separate the different
parts of the command string) and a colon (to separate the machine
address from the port.)

As an example, if you have 3 computers with IP addresses,
shack1.myhouse.org, shack2.myhouse.org, shack3.myhouse.org. You could
decide to run ncat on port 12001 on all of them.

The trlog version on shack1.myhouse.org would have the line
\begin{verbatim}
MULTI PORT = SERIAL ncat;12001;shack2.myhouse.org:12001
\end{verbatim}

The trlog version on shack2.myhouse.org would have the line
\begin{verbatim}
MULTI PORT = SERIAL ncat;12001;shack3.myhouse.org:12001
\end{verbatim}

The trlog version on shack3.myhouse.org would have the line
\begin{verbatim}
MULTI PORT = SERIAL ncat;12001;shack1.myhouse.org:12001
\end{verbatim}

\section{xterm setup}
The xterm terminal emulator\index{xterm set up}
for X is distributed with every linux system,
so I decided to run TRlog inside an xterm.
TRlog expects VGA sized screens, so you need to set up the xterm with
the correct options. These are most easily set in the .Xresources
file\index{.Xresources file}
in your home directory (notice the ``.'' at the beginning of the
file name!).  Here is an example which is included in the
distribution in trlinux-rN.NN/files/Xresources:
\begin{verbatim}
!Use:
!xterm -class DosTerm
DosTerm*faceName: Lucida
DosTerm*faceSize: 14
DosTerm*renderFont: true
DosTerm*VT100*background: black
DosTerm*VT100*foreground: white
DosTerm*VT100*geometry: 80x25
DosTerm*eightBitInput: false
DosTerm*VT100.translations: #override \
  Ctrl<Key>minus: string(0x1f)


!Use:
!xterm -class BigDosTerm
BigDosTerm*faceName: Liberation Mono
BigDosTerm*faceSize: 12
BigDosTerm*VT100*background: black
BigDosTerm*VT100*foreground: white
BigDosTerm*VT100*geometry: 80x50
BigDosTerm*eightBitInput: false
BigDosTerm*VT100.translations: #override \
  Ctrl<Key>minus: string(0x1f)
\end{verbatim}

You can then start these with the commands
\begin{verbatim}
xterm -class DosTerm
\end{verbatim}
or
\begin{verbatim}
xterm -class BigDosTerm
\end{verbatim}
You will want to use the BigDosTerm class if you have a band map  or visible
dupe sheet.
You can use the command
\begin{verbatim}
fc-list :scalable=true:spacing=mono: family
\end{verbatim}
to list font names\index{Changing the font}.
A name before the first colon (:) can be used
in place of my choices above for the faceName values. The faceSize
values can be changed to change the size of the font and the xterm
window\index{Changing the font size}.

{\em It is absolutely necessary that
the DosTerm be 80 columns and 25 rows,
and the BigDosTerm 80 columns and 50 rows. }\index{xterm sizes}
You can check this with the commands
\begin{verbatim}
set |grep LINES
set |grep COLUMNS
\end{verbatim}
the first should print LINES=80 and the second LINES=25 or LINES=50 depending
on whether you are in a DosTerm or a BigDosTerm. If these do not
match, you likely have a font that is too big to allow the xterm its
full size. In that case change the faceSize value in the .Xresources file.

All of this can be a annoying to get set up the way you like, but at
least it only needs to be done once.

\section{File location}
Since under DOS everyone who touched a computer had complete control, DOS
TR would write various files back to the directory which held the TR and POST
executables. All modern operating systems are multiuser. Therefore,
you should be able to have one copy of trlog installed on your computer,
and all users of the computer have access to the executables, and contest
data, but should not be able to change each other's files, and only
the root user should be able to change the installation.

The linux version starting with version 0.16, will create a .trlog subdirectory
in your home directory if needed, i.e. it will create \$HOME/.trlog if
needed\index{.trlog directory}. This will be where changes to
the files CABNAME.DAT, ADDRESS.DAT, TRMASTER.DTA will be stored for each
user. This means that if you have a guest operator, you can make a new
user account for that operator, and changes made by that user should not
affect any of your configuration files. You should be able to install
trlinux in any convenient directory e.g. /usr/local/trlinux which
can be globally readable and executable, but not writeable.

\section{YCCC SO2R box}
Information about the YCCC SO2R box\index{YCCC SO2R box},
its design, and how to obtain
one, can be found at
\url{<http://k1xm.org/SO2R/index.html>}.

After assembling your SO2R box, it is highly recommended that you first
test that it functions correctly on a Windows computer, so that you
know the SO2R box is working.

\subsection{Setting permissions}
On linux the first thing to do is to set up permissions for the YCCC SO2R box.
Trlog must have
have read/write permission to the SO2R box hardware
device\index{YCCC SO2R box!device permissions}.
The standard way to set this up is to add a udev\index{udev rules} rule,
so when the operating
system detects the YCCC SO2R box, it will set the desired owner, group,
and permissions for the YCCC SO2R box device.
\begin{itemize}
\item
As superuser, use a text editor to create a new rule file. I use
\begin{verbatim}
/etc/udev/rules.d/50-usb-so2r.rules
\end{verbatim}
\item
The contents of this file should tell the operating system what you want
to do. Here are two examples:
\begin{itemize}
\item
With contents
\begin{verbatim}
SUBSYSTEM=="usb", ATTR{idVendor}=="16c0", ATTR{idProduct}=="065e", MODE="0666"
\end{verbatim}
all users will have read/write permission. If your computer is secure,
this should work fine.
\item

A more secure solution uses instead a rules file
with contents
\begin{verbatim}
SUBSYSTEM=="usb", ATTR{idVendor}=="16c0", ATTR{idProduct}=="065e", MODE="0660"
,GROUP="so2rbox"
\end{verbatim}
all on one line.
Only members of the so2rbox group will have read/write permission.
You will have to create this group. You can do this
using the command groupadd.
That is on Slackware, with my
user name w9cf I can add the group with
\begin{verbatim}
sudo groupadd so2rbox
\end{verbatim}
and then add my account to that group using
\begin{verbatim}
sudo gpasswd -a w9cf so2rbox
\end{verbatim}
You need to log out, and log back in, in order to have your login session
reread the group file. Make sure that typing
\begin{verbatim}
groups
\end{verbatim}
at the command line returns the name so2rbox (or whatever group name you
used.)
\end{itemize}
\end{itemize}
In the udev rules above,
the hex vendor and product IDs are those reported by the
firmware of the YCCC SO2R box. All YCCC SO2R boxes use these same values.

\subsection{Optional: Checking YCCC SO2R box permissions}
You can double check that the udev rule is working and
giving the proper permissions. If your linux works like mine:
\begin{itemize}
\item
Type
\begin{verbatim}
lsusb -d 16c0:065e
\end{verbatim}
You can also type lsusb without options,
but you will get a listing of all your usb
devices not just the YCCC SO2R box.
\item
You should see something like
\begin{verbatim}
Bus 002 Device 003: ID 16c0:065e VOTI
\end{verbatim}
where VOTI is the manufacturer ID of the YCCC box. Your bus and device
numbers will likely be different.
\item
Check the permissions on this device by using a command like
\begin{verbatim}
ls -l /dev/bus/usb/002/003
\end{verbatim}
where you should use the Bus and Device numbers from your lsusb command
output instead of 002 and 003 above.
The first example udev rule above will show something like
\begin{verbatim}
crw-rw-rw- 1 root root 189, 1 Sep  4 21:44 /dev/bus/usb/002/003
\end{verbatim}
where the crw-rw-rw shows a character device owned by root with group root
but all users have read/write permission.
The second example udev rule above will show something like
\begin{verbatim}
crw-rw---- 1 root so2rbox 189, 1 Sep  4 21:44 /dev/bus/usb/002/003
\end{verbatim}
which has owner and group read/write permission and the group is set
to so2rbox. Users not in the so2rbox group do not have
permission to communicate with the
YCCC SO2R box.
\end{itemize}

\subsection{Using the YCCC SO2R box}
To use the box, include the configuration line\index{YCCC SO2R BOX ENABLE}
\begin{verbatim}
yccc so2r box enable = true
\end{verbatim}
in LOGCFG.DAT

Other LOGCFG.DAT commands that can be used
are\index{SO2R MICROPHONE RELAY ENABLE}
\begin{verbatim}
so2r microphone relay enable = true
\end{verbatim}
with value true or false. This turns the microphone relays on or off
in the YCCC SO2R box. Default is true.

\begin{verbatim}
so2r blend enable = true
\end{verbatim}
with\index{SO2R BLEND ENABLE}
value true or false. Default is false. This turns on blending (i.e.
some of the left radio in the right ear, and some of the right radio
in the left ear).

\begin{verbatim}
so2r blend = 50
\end{verbatim}
where\index{SO2R BLEND}
the value should be between 0 (for minimal blending) to 255
(for maximal blending).

\begin{verbatim}
so2r headphone mode = normal
\end{verbatim}
The other values\index{
SO2R HEADPHONE MODE
} are spatial and symmetric. See the YCCC SO2R box documentation
for details. Default is normal.


In latch mode the transmitting rig is normally muted in the headphones.
If the headphone mode is normal (stereo), the nontransmitting rig
will be in both ears. On cw, if the paddle or keyboard cw is used, the
transmitting rig is not muted. You can turn on latch mode with
\begin{verbatim}
so2r latch = true
\end{verbatim}
\index{SO2R LATCH}

\begin{verbatim}
so2r rig1 map = 1
so2r rig2 map = 2
\end{verbatim}
\index{SO2R RIG1 MAP}\index{SO2R RIG2 MAP}

These set the mapping from rig1 and rig2 to the output 1 through 4 on
the back of the YCCC box. A value of 0 will use whatever the box is
set up for currently. Default is 0.
You can also use the values -1, -2, -3, -4. These will save the mapping
in the YCCC box's EEPROM. The EEPROM values are loaded on power up.

You can change all of these value using the control-J menu.

The second way to send commands to the SO2R box\index{YCCC SO2R box!message
commands} is to
include commands as messages analogously to the way that
raw rig control commands can be used as messages, i.e.
$\langle$03$\rangle$SO2R=Command$\langle$04$\rangle$.
The $\langle$03$\rangle$ and $\langle$04$\rangle$  are CTRL-C and CTRL-D.
The current commands are:
\begin{verbatim}
LATCHON
LATCHOFF
LATCHTOGGLE
RX1
RX2
STEREO
RXA
RXI
\end{verbatim}
RX1\index{RX1}\index{RX2}\index{STEREO}\index{RXA}\index{RXI}
puts receive focus on Rig 1, RX2 puts receive focus on Rig 2.
RXA puts receive focus on the active radio.
RXI puts receive focus on the inactive radio.
STEREO puts the SO2R box in Stereo mode. 

TR puts the band data\index{YCCC SO2R box!band data}
on the auxiliary connector (the DB25 connector)
of the SO2R box. Pins 1-4 are the band data for rig 1, Pins 5-8 are
the band data for rig 2. Pin 9 of the axiliary port is the common ground.
These set such that Aux pins 1, 2, 3, 4,
are equivalent to the parallel port pins 2, 7, 8, 9, respectively, of DOS TR,
and Aux pins 5, 6, 7, 8, are equivalent to pins 2, 7, 8, 9 of rig 2's
band output parallel port.

You can plug a footswitch\index{YCCC SO2R box!footswitch}
into the PTT connector of the YCCC SO2R box.
It then becomes the footswitch with operation as explained in the DOS TR
manual. Except,
\begin{itemize}
\item
The footswitch will always toggle the radio PTT line
on the rig connectors of the YCCC box, in all modes. This is the way the
hardware is set up, and cannot be changed (except by modifying the
YCCC SO2R box firmware.) This means that you can only use the
SO2R box footswitch for other operations if you do not connect the
connector PTT line to the rig.
\item
CW grant mode does not work. Footswitch mode Normal is untested.
\end{itemize}

\subsection{Controlling a digital voice keyer}
Pins 18-24 on the YCCC SO2R box DB-25 can be used in place of a
parallel port to control a digital voice keyer. Pin 18 is DVK0 and is
pulsed to abort the voice keyer. Pins 19-24 are connected to DVK1-DVK6.
To turn this on use
\begin{verbatim}
DVK PORT = YCCC
\end{verbatim}
The operation is the same as the W9XT DVK as explained in the DOS TR
manual. Recording from TR commands is not supported.

Normally TR is set up to trigger recording on a W9XT DVK card when the
corresponding control keys are pressed. You can disable this behavior,
so that control-f1 through control-f10 can be used for other messages
with the command
\begin{verbatim}
DVK CONTROL KEY RECORD = FALSE
\end{verbatim}
This can also be changed in the control-j menu.

\section{Using a rig's internal voice keyer}
You can use a rig's internal voice as the DVK by first setting
\begin{verbatim}
dvk radio enable = true
\end{verbatim}
You must also set up the strings to send as
rig commands to trigger the DVK messages using
\begin{verbatim}
dvk radio1 stop cmd =
dvk radio1 play dvk1 cmd =
dvk radio1 play dvk2 cmd =
dvk radio1 play dvk3 cmd =
dvk radio1 play dvk4 cmd =
dvk radio1 play dvk5 cmd =
dvk radio1 play dvk6 cmd =

dvk radio2 stop cmd =
dvk radio2 play dvk1 cmd =
dvk radio2 play dvk2 cmd =
dvk radio2 play dvk3 cmd =
dvk radio2 play dvk4 cmd =
dvk radio2 play dvk5 cmd =
dvk radio2 play dvk6 cmd =
\end{verbatim}
These are then accessed in the funciton key messages as for a external
DVK with something like
\begin{verbatim}
cq ssb memory f1 = dvk1
\end{verbatim}

\section{Using rigctld}
Elecraft/Kenwood and Icom rigs are supported natively. For other rigs,
you need to install the hamlib library which comes with the rigctld
daemon. You can then specify the rig with a configuration command like:
\begin{verbatim}
radio one type = rigctld;4532;/dev/ttyUSB0;1608;57600;38;
radio two type = rigctld;4534;/dev/ttyS0;221;4800;38;
\end{verbatim}

Starting the radio type with rigctld tells TR to start the rigctld
program for that radio. The semicolon separated values following are
the port to use (these need to be different if more than one rigctld is
started), the serial port that connects to the radio, the hamlib
radio model number,
the baud rate, and finally the CIV address.  A list of radio model numbers
supported by hamlib can be found by typing
\begin{verbatim}
rigctl -l
\end{verbatim}
at the linux command-line prompt in a terminal window. You can usually
find the model number quickly with something like the examples below
\begin{verbatim}
rigctl -l 2>/dev/null |grep -i k2
rigctl -l 2>/dev/null |grep -i orion
rigctl -l 2>/dev/null |grep -i kx3
rigctl -l 2>/dev/null |grep -i ic-7700
\end{verbatim}
The CIV address only matters for Icom rigs, and should match the rig's
CIV address. For other brands, any value will work.

The hamlib recommended rigctld port are the even numbered ports
4532, 4534, 4536, etc. You can,
of course, use any unused nonprivileged port.

\section{Real time contest score reporting}
TR can report contest scores and band breakdowns to one of the contest
scoreboards if desired. Currently, this is not user friendly at all.
It must be set up in the LOGCFG.DAT
or equivalent file at program start up. Equivalent control-J commands have
not been implemented.  {\em Currently no input checking is enforced,
and correct input defaults are not set up for any contests. Everything
must be set manually.}

The commands are described in the next sections:

\subsection{Enabling and specifying the score reporting web site} These
commands choose the score reporting web site, and sets your credentials,
if any.

\begin{itemize}
\item
\begin{verbatim}
score report enable =
\end{verbatim}
Setting this to true
turns on score reporting.
False is the default.
\item
\begin{verbatim}
score report post url =
\end{verbatim}
This is the url for the web site used for score reporting. The
current sites are
\url{https://contestonlinescore.com/post/}
and
\url{https://cqcontest.net/postscore.php}. There are also
corresponding unencrypted http sites
\url{http://post.contestonlinescore.com/}
and
\url{http://cqcontest.net/postscore.php}.

\item
\begin{verbatim}
score report username =
\end{verbatim}
The user name used for authentication at the score reporting web site.
This is typically your call sign. This is only used if
the url begins with ``https:''. The default is the null string.

\item
\begin{verbatim}
score report password = 
\end{verbatim}
The password used for authentication at the score reporting web site.
This is only used if the url begins with ``https:''.
The default is the null string.
Please note that since your username and password will be stored {\em
in the clear} in the TR configuration file, that this is not
secure.

You should make the file containing your password readable only by you, i.e.
with a command like
\begin{verbatim}
chmod 600 LOGCFG.DAT
\end{verbatim}
However,
any administrator on your computer, or any malware that gains root permission
would be able to read this trivially.
You should think of this username password as locking your front door
and then leaving a key under the door mat or under a nearby flower pot.
It tells people that you don't want them to barge in on you, but you know
a burglar will be able to get in. 

Making this secure does not seem like a good use of developer time, since the
only consequences of stolen credentials is that someone could post wrong
scores from you until you notified the score reporting
site to reset your account. If this bothers you, use the unsecured web sites
that do not require a username and password.

{\em Be sure that the password you use here
is not used for any important account.}
\end{itemize}

\subsection{Operator entry class}
The following commands speficy your entry class. The default for all of
these is a null string.
\begin{itemize}
\item
\begin{verbatim}
score report class ops =
\end{verbatim}
The choices are SINGLE-OP, MULTI-OP, or CHECKLOG.

\item
\begin{verbatim}
score report class bands =
\end{verbatim}
The choices are ALL, 160M, 80M, 40M, 20M, 15M, 10M, 6M, 2M.

\item
\begin{verbatim}
score report class power =
\end{verbatim}
The choices are HIGH, LOW, QRP.

\item
\begin{verbatim}
score report class transmitter =
\end{verbatim}
The choices are ONE, TWO, LIMITED, UNLIMITED, SWL.

\item
\begin{verbatim}
score report class mode =
\end{verbatim}
The choices are MIXED, CW, SSB, RTTY

\begin{verbatim}
score report class assisted =
\end{verbatim}
The choices are ASSISTED, and NON-ASSISTED.
\end{itemize}

\subsection{Contest spefifics}
\begin{itemize}
\item
\begin{verbatim}
score report breakdown enable =
\end{verbatim}
Setting this to true uploads band and mode breakdowns for QSOs and multipliers,
otherwise just the total QSOs, multipliers,  QSO points and score are reported.
False is the default.
\item
\begin{verbatim}
score report contest =
\end{verbatim}
The is the contest name as specified by the score reporting web site.
This does not appear to be standardized. It is usually the same string
that is used by the Cabrillo file, but not always.
One list can be found here \url{https://contestonlinescore.com/settings/}.
\end{itemize}


\subsection{Multipliers}
Assigning the 4 kinds of TR multipliers to the score reporting names must
also be done manually. These are done using
\begin{verbatim}
score report dx multipliers =
score report zone multipliers =
score report prefix multipliers =
score report domestic multipliers =
\end{verbatim}

Only the multipliers used for your particular
contest must be set. If the same name is used for more than
one type of multiplier (for example
in NA Sprint, I think both {\bf score report domestic multipliers} and
{\bf  score report dx multipliers} should be set to {\bf state}), the
multipliers are summed in the upload report.
Multipliers not used by TR for the contest are ignored. The typical values
from \url{https://contestonlinescore.com/settings/}.
are {\bf country}, {\bf state}, {\bf zone}, {\bf hq}, {\bf gridsquare}.

So for CQ WW you would use
\begin{verbatim}
score report dx multipliers = country
score report zone multipliers = zone
\end{verbatim}
while for IARU you would use
\begin{verbatim}
score report domestic multipliers = hq
score report zone multipliers = zone
\end{verbatim}

\subsection{Miscellaneous score reporting commands}
\begin{itemize}
\item
\begin{verbatim}
score report club =
\end{verbatim}
Set the name of your club in your score report. The default is null.

\item
\begin{verbatim}
score report interval minutes =
\end{verbatim}
Any positive integer. This gives the approximate time between score uploads
in minutes. The default is 2. The first score report is sent at about
1/20 of this interval after the program starts. So after 6 seconds for
the default.
\item
\begin{verbatim}
score report debug enable =
\end{verbatim}
Setting this to true turns on verbose debugging output which is sent to
the file curl.dbg in the start up directory. This can be useful if something
doesn't work. It shows the xml data that will be uploaded, followed by
libcurl verbose responses.
False is the default.

If you want to make some quick tests without uploading data, simply redirect
the uploading. For example, set
\begin{verbatim}
score report post url = http://localhost:5555
\end{verbatim}
which sends the output to port 5555 on your computer.
Then, in a terminal window start:
\begin{verbatim}
ncat -l -p 5555
\end{verbatim}
and
you will see the transmitted output of TR. Note in this case
since ncat is not
a real http server, it will not give a response so the transaction will
not complete.

\end{itemize}
\subsection{Examples}
These were two contests that allowed uploading of data while the code
was being developed.
The Russian world-wide multimode contest:
\begin{verbatim}
score report enable = true
score report debug enable = false
score report breakdown enable = true
score report contest = RUS-WW-MM
score report class ops = SINGLE-OP
score report class bands = ALL
score report class power = LOW
score report class transmitter = ONE
score report class mode = CW
score report class assisted = NON-ASSISTED
score report post url = https://contestonlinescore.com/post/
score report username = w9cf
score report password = secretpassword
score report club = Arizona Outlaws Contest Club
score report interval minutes = 1
score report domestic multipliers = none
score report dx multipliers = country
score report zone multipliers = state
score report prefix multipliers = none
\end{verbatim}
and the Scandinavian activity contest SSB
\begin{verbatim}
score report enable = true
score report debug enable = false
score report breakdown enable = true
score report contest = SAC-SSB
score report class ops = SINGLE-OP
score report class bands = ALL
score report class power = LOW
score report class transmitter = ONE
score report class mode = SSB
score report class assisted = NON-ASSISTED
score report post url = https://cqcontest.net/postscore.php
score report username = w9cf
score report password = secretpassword
score report club = Arizona Outlaws Contest Club
score report interval minutes = 1
score report domestic multipliers = none
score report dx multipliers = country
score report zone multipliers = none
score report prefix multipliers = country
\end{verbatim}


\section{Troubleshooting}
\subsection{Some keys or key combinations don't work}
This is likely either some program like a window manager or helper
is intercepting the key presses before reaching the xterm, or your xterm
version is not sending the standard function key strings.

You can test these as follows:
\begin{itemize}
\item
It will be convenient to copy the Xresources and xinitrc files
from the files subdirectory of the trlog distribution to you home directory.
\item
Exit X. If you are running a display manager (i.e. you logon through a
pretty X display rather than the console login prompt:
\begin{verbatim}
login:
\end{verbatim}
you need to do something like:
\begin{verbatim}
telinit 3
\end{verbatim}
as root to enter run level 3 which often is the console login. However,
some distributions make run levels 3 through 5 identical -- see your
distributions documentation and/or check
the comments in /etc/inittab if this doesn't work.
\item
Log in at the console prompt. Type
\begin{verbatim}
ls -l Xresources
ls -l xinitrc
\end{verbatim}
to make sure those files are available.
Now run a minimal X.
\begin{verbatim}
xinit xinitrc
\end{verbatim}
which is X with no window manager and only a single xterm. In this xterm
you can try running trlog and see if the function keys work.
\item
You can also simply type
\begin{verbatim}
cat
\end{verbatim}
in the xterm and press the key combinations that aren't working in trlog.
The strings that the xterm sends will be displayed (escape is \verb|^|[ ).
Compare these to the expected strings in xterm.txt in the files subdirectory.

\item
If trlog works in this minimal xterm,
then some part of your X installation is capturing the
keystrokes before handing them to the xterm.
You will have to figure out what this is, and turn it off when running
trlog.
Usually it is the window manager, but other helper programs can capture
keystrokes too.
\item
If your version of xterm is not sending the correct strings expected
by trlog, you can
use the xterm translations property to change them. 

\end{itemize}
\subsection{Shared libraries not found}
Since this is Tree's source code, he makes the rules.
I therefore am only distributing a binary version.

To legally distribute a binary only version of trlog, some libraries
(e.g. libsndfile) can only be dynamically linked. Here is the output
of ldd trlog on my system:\index{Shared libraries used}
\begin{verbatim}
        linux-gate.so.1 =>  (0xffffe000)
        libX11.so.6 => /usr/lib/libX11.so.6 (0xb75e2000)
        libasound.so.2 => /usr/lib/libasound.so.2 (0xb7515000)
        libieee1284.so.3 => /usr/lib/libieee1284.so.3 (0xb750b000)
        libpthread.so.0 => /lib/libpthread.so.0 (0xb74f1000)
        libsndfile.so.1 => /usr/lib/libsndfile.so.1 (0xb748b000)
        libdl.so.2 => /lib/libdl.so.2 (0xb7487000)
        libusb-1.0.so.0 => /usr/local/lib/libusb-1.0.so.0 (0xb747b000)
        libc.so.6 => /lib/libc.so.6 (0xb7318000)
        librt.so.1 => /lib/librt.so.1 (0xb730f000)
        libxcb.so.1 => /usr/lib/libxcb.so.1 (0xb72f6000)
        libXau.so.6 => /usr/lib/libXau.so.6 (0xb72f3000)
        libXdmcp.so.6 => /usr/lib/libXdmcp.so.6 (0xb72ee000)
        libm.so.6 => /lib/libm.so.6 (0xb72c8000)
        /lib/ld-linux.so.2 (0xb772c000)
        libFLAC.so.8 => /usr/lib/libFLAC.so.8 (0xb7279000)
        libvorbisenc.so.2 => /usr/lib/libvorbisenc.so.2 (0xb7102000)
        libvorbis.so.0 => /usr/lib/libvorbis.so.0 (0xb70db000)
        libogg.so.0 => /usr/lib/libogg.so.0 (0xb70d5000)
\end{verbatim}
Check your versions of these libraries if there are problems on your
system. These libraries are mostly
used for parallel port, usb, and sound card access.
The trlog executable is linked with an RPATH that points
to the lib subdirectory in the same directory as the trlog executable.
If your system uses default libraries that differ from the ones
above and do not work, you can put alternative libraries in the lib
subdirectory. You should then see them dynamically linked
if you run ldd trlog. The Slackware 13.37.0 libraries work on my
system.


\section{Reporting a program crash}
Nothing that you as a user do should cause the program to crash.
The most likely cause of a program crash is a bug in the program.

Hardware problems (like computer memory that is failing) are a possible,
but much
less likely reason for a program crash.
Usually, if you are having hardware problems,
multiple programs will crash intermittently.

Starting with version 0.32, the linux version of trlog will try to produce
a core dump file\index{core files}
if it crashes with an illegal instruction, floating
point exception, or segmentation fault. Most linux distributions are
set to not allow core dumps by default. If you are experiencing program
crashes, and are willing to help me find the problem, please do the following:
\begin{itemize}
\item
Start the xterm you use for trlog.
\item
In this xterm, before starting trlog, type the command
\begin{verbatim}
ulimit -c
\end{verbatim}
If this reports anything but ``unlimited'', type
\begin{verbatim}
ulimit -c unlimited
\end{verbatim}
\item
Optional:
you
can check that trlog can now produce a core file. Start trlog, and in
another terminal type
\begin{verbatim}
killall -s 11 trlog
\end{verbatim}
This should cause trlog to crash and create a core file (usually named
``core'') in the current directory.
Now remove this file
\begin{verbatim}
rm core
\end{verbatim}
If the core file is not produced, it is possible that you have a newer
linux kernel (3.7 or later) that has been configured without the
CONFIG\_COREDUMP option. If this is the case, you will not be able to
produce core dumps unless you recompile the kernel with this option turned
on.
\item
When trlog crashes, send me an email with the following attachments
\begin{itemize}
\item
The core file.
\item
Your configuration file(s) LOGCFG.DAT or equivalent.
\item
Any information on what action you think might have led to the crash.
\end{itemize}
\end{itemize}

\section{Using pseudoterminals -- Deprecated}
{\em This section is now deprecated.} You should use the shell device
for packet and the ncat device for the multi-op network. The pseudoterminal
support does still work.

You can set up packet, mulit-op networking etc.
exactly as under DOS trlog using
serial ports.

An alternative that allows you to connect using standard networking
connections -- so, for example, you can connect to a packet cluster
over telnet, is to use pseudoterminals.
You can use either the legacy berkeley pseudoterminals if you have
support compiled in your kernel or as of
version 0.04, standard pseudottys using /dev/ptmx etc.
Note the legacy Berkeley psudottys have been deprecated, and may
not be compiled into your kernal by default.  Now that standard
pseudottys are supported in trlog, you should probably use them.

\subsection{Telnet packet with standard pseudottys}
Here's a minimal LOGCFG.DAT to use a pseudotty for telnet packet:
\begin{verbatim}
MY CALL = W9CF
CONTEST = CQ WW
DISPLAY MODE = COLOR
PACKET PORT = SERIAL ptypacket
EIGHT BIT PACKET PORT = TRUE
BAND MAP ENABLE = TRUE
\end{verbatim}

You can choose any unique name that starts with pty or PTY.  I chose
the name ptypacket above. Trlog will
open the next available pseudo tty master and use it for the serial
port. It then opens the file given by the name you used (here ptypacket)
and write the slave device name to that file.

I use the following script to fire up trlog (trlog must be in your path)
\begin{verbatim}
#!/bin/bash
xterm -class BigDosTerm -e trlog &
sleep 2
ptyslave=`cat ptypacket`
stty icrnl erase '^h' < /dev/ttyqa
telnet n7az.net < $ptyslave > $ptyslave
\end{verbatim}
The first line creates an xterminal of the BigDosTerm class as described
above, and runs trlog putting the process in the background.
It then sleeps 2 seconds to allow trlog to set up the master end of the
pseudotty. The next line reads the slave device from the
file ptypacket that trlog creates. Then it sets up the slave end translating
carriage returns to line feeds and using ctrl-h as the erase character.
The last line starts a telnet connection to the n7az.net telnet cluster.

When trlog opens, you type ctrl-b to get to the packet window, and login
to n7az.net. You can type sh/dx to get some spots to start. Close the
packet window with another ctrl-b and it should act exactly as a serial
port connection to a packet radio cluster.

\subsection{Telnet packet with legacy Berkeley pseudotty}
Here's a minimal LOGCFG.DAT to use a legacy pseudotty for telnet packet:
\begin{verbatim}
MY CALL = W9CF
CONTEST = CQ WW
DISPLAY MODE = COLOR
PACKET PORT = SERIAL /dev/ptyqa
EIGHT BIT PACKET PORT = TRUE
BAND MAP ENABLE = TRUE
\end{verbatim}
where I have assumed that no other process is using the master /dev/ptyqa.
Normally this will be true since most linux codes will use the more
modern /dev/ptmx method, but if not, look in /dev and choose a different
pty device. Note, you need to have the required permissions to open the
device. On my systems, they are owned by root and in the tty group, so
including my username in /etc/group for the tty group gives me permission.
You can do this with the command
\begin{verbatim}
sudo gpasswd -a w9cf tty
\end{verbatim}
You must log out and log back in for this to take effect. You can check
the groups you are a member of with the command
\begin{verbatim}
groups
\end{verbatim}

I use the following script to fire up trlog (trlog must be in your path)
\begin{verbatim}
#!/bin/bash
xterm -class BigDosTerm -e trlog &
sleep 2
stty icrnl erase '^h' < /dev/ttyqa
telnet n7az.net < /dev/ttyqa > /dev/ttyqa
\end{verbatim}
The first line creates an xterminal of the BigDosTerm class as described
above, and runs trlog putting the process in the background.
It then sleeps 2 seconds to allow trlog to set up the master end of the
pseudotty. The next line says to set up the slave end translating
carriage returns to line feeds and using ctrl-h as the erase character.
The last line starts a telnet connection to the n7az.net telnet cluster.

When trlog opens, you type ctrl-b to get to the packet window, and login
to n7az.net. You can type sh/dx to get some spots to start. Close the
packet window with another ctrl-b and it should act exactly as a serial
port connection to a packet radio cluster.

\subsection{Multi networking with standard pseudottys}
The N6TR multi networking can be set up the same way on two machines
with the configuration files below. Assume the first machine has ip
address 192.168.0.1 with TRlog configuration file
\begin{verbatim}
MY CALL = W9CF
CONTEST = CQ WW
DISPLAY MODE = COLOR
MULTI PORT = SERIAL ptymulti
COMPUTER ID = X
\end{verbatim}
and the second with ip address 192.168.0.2 and configuration file
\begin{verbatim}
MY CALL = W9CF
CONTEST = CQ WW
DISPLAY MODE = COLOR
MULTI PORT = SERIAL ptymulti
COMPUTER ID = Y
\end{verbatim}
It is convenient (but not necessary)
to use the same pty name (here I chose ptymulti) for all
of the machines.

Start trlog on each machine. Note the current
directory name where the LOGCFG.DAT
file. Below I will assume this is \$HOME/radio/multi.
In this example,
trlog will connect will write out the name of the slave device
in the file \$HOME/radio/multi/ptymulti on each machine.
Once all the trlog instances are running, you can connect the
slave devices together.
On machine 192.168.0.1, execute the script in another terminal window
\begin{verbatim}
#!/bin/sh
nextcomputer=192.168.0.2
nextptyfile=radio/multi/ptypacket
mypty=`cat ptypacket`
ssh $nextcomputer "cat > \`cat $nextptyfile\`" < $mypty
\end{verbatim}
and similarly on 192.168.0.2 execute
\begin{verbatim}
nextcomputer=192.168.0.1
nextptyfile=radio/multi/ptypacket
mypty=`cat ptypacket`
ssh $nextcomputer "cat > \`cat $nextptyfile\`" < $mypty
\end{verbatim}
For more computers,
each will need to execute such a command to connect to the
next machine in the circular N6TR multi network simulating a serial
port connections around the loop. That is for four computers,
computer 1 connects to computer 2
which connects to computer 3, which connects to computer 4, which connects
back to computer 1.

\subsection{Multi networking with legacy pseudottys}
The N6TR multi networking can be set up the same way on two machines
with the configuration files below. Assume the first machine has ip
address 192.168.0.1 with TRlog configuration file
\begin{verbatim}
MY CALL = W9CF
CONTEST = CQ WW
DISPLAY MODE = COLOR
MULTI PORT = SERIAL /dev/ptyqb
COMPUTER ID = X
\end{verbatim}
and the second with ip address 192.168.0.2 and configuration file
\begin{verbatim}
MY CALL = W9CF
CONTEST = CQ WW
DISPLAY MODE = COLOR
MULTI PORT = SERIAL /dev/ptyqb
COMPUTER ID = Y
\end{verbatim}
It is convenient (but not necessary)
to use the same master pseudotty device name for all
of the machines.

Start trlog on each machine. This will open all the master pseudottys.
Now on machine 192.168.0.1, execute the command in another terminal window
\begin{verbatim}
ssh 192.168.0.2 "cat > /dev/ttyqb" < /dev/ttyqb
\end{verbatim}
and similarly on 192.168.0.2 execute
\begin{verbatim}
ssh 192.168.0.1 "cat > /dev/ttyqb" < /dev/ttyqb
\end{verbatim}
For more computers,
each will need to execute such a command to connect to the
next machine in the circular N6TR multi network simulating a serial
port connections around the loop. That is for four computers,
computer 1 connects to computer 2
which connects to computer 3, which connects to computer 4, which connects
back to computer 1.

\section{Disclaimer}
\label{disclaim}

THIS SOFTWARE IS PROVIDED BY THE AUTHOR AND CONTRIBUTORS ``AS IS'' AND
ANY EXPRESS OR IMPLIED WARRANTIES, INCLUDING, BUT NOT LIMITED TO, THE
IMPLIED WARRANTIES OF MERCHANTABILITY AND FITNESS FOR A PARTICULAR PURPOSE
ARE DISCLAIMED.  IN NO EVENT SHALL THE AUTHOR OR CONTRIBUTORS BE LIABLE
FOR ANY DIRECT, INDIRECT, INCIDENTAL, SPECIAL, EXEMPLARY, OR CONSEQUENTIAL
DAMAGES (INCLUDING, BUT NOT LIMITED TO, PROCUREMENT OF SUBSTITUTE GOODS
OR SERVICES; LOSS OF USE, DATA, OR PROFITS; OR BUSINESS INTERRUPTION)
HOWEVER CAUSED AND ON ANY THEORY OF LIABILITY, WHETHER IN CONTRACT, STRICT
LIABILITY, OR TORT (INCLUDING NEGLIGENCE OR OTHERWISE) ARISING IN ANY WAY
OUT OF THE USE OF THIS SOFTWARE, EVEN IF ADVISED OF THE POSSIBILITY OF
SUCH DAMAGE.

\newpage
\printindex
\end{document}
